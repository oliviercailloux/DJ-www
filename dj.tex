\RequirePackage[l2tabu, orthodox]{nag}
\documentclass[french, english]{beamer}
\input{preamble/packages}
\input{preamble/redac}
\input{preamble/math_basics}
\input{preamble/math_mine}
\input{preamble/draw}

\title[Deliberated judgment]{\glspdffmtfull{DJ}: what, why, how?}
\subject{Preference modeling}
\keywords{Arguments}
\author{Olivier Cailloux \inst{1}}
\institute[LAMSADE]{\inst{1} LAMSADE, Université Paris-Dauphine}
\date{\formatdate{7}{11}{2024}}

\begin{document}
\begin{frame}[plain]
	\tikz[remember picture,overlay]{
		\path (current page.south west) node[anchor=south west, inner sep=0] {
			\includegraphics[height=8mm]{Dauphine-Noir.png}
		};
		\path (current page.south east) node[anchor=south east, inner sep=0] {
			\includegraphics[height=1cm]{LAMSADE95.jpg}
		};
		\path (current page.south) ++ (0, 4em) node[anchor=south, inner sep=0] {
			\scriptsize\textcolor{blue}{\url{https://github.com/oliviercailloux/DJ-www}}
		};
	}
	\titlepage
\end{frame}
\addtocounter{framenumber}{-1}

\begin{frame}
	\frametitle{Outline}
	\tableofcontents[hideallsubsections, sectionstyle=shaded/show]
\end{frame}

\AtBeginSection{
	\begin{frame}
		\frametitle{Outline}
		\tableofcontents[currentsection, hideallsubsections]
	\end{frame}
}

\glsunset{DJ}

\section{What?}
\begin{frame}
	\frametitle{\glsfmtlong{DJ}}
	\begin{itemize}
		\item \emph{Someone}’s \gls{DJ} \emph{about an issue} facing a given set of \emph{arguments}: her judgment after considering the arguments
		\item Example: your judgment about nuclear power after considering some arguments given by Greenpeace and EDF
		\item Intuition: judgment may change after learning something
		\item … but may change back after learning more!
		\item Need systematic method to study \glspl{DJ}
	\end{itemize}
\end{frame}

\begin{frame}
	\frametitle{Deliberated preference}
	\begin{itemize}
		\item Someone’s deliberated preference about an issue facing arguments given possible courses of action (or attitudes): her relative positioning of the possibilities, after considering the arguments
		\item Deliberated preference: relative positioning of possible judgments
		\item Typically, represents how the person chooses
		\item Example: whether you prefer nuclear power to solar power (… to coal power, …), after considering some arguments given by Greenpeace and EDF
	\end{itemize}
\end{frame}

\section{Why?}
\begin{frame}
	\frametitle{Contrast with shallow preference}
	\begin{itemize}
		\item Economists study preferences
		\item To model choice
		\item Sometimes used normatively
		\item People prefer this to that, therefore, we should orient society towards…
	\end{itemize}
  Similarity: the individual’s position (no a priori position on the quality of the arguments)
\end{frame}

\begin{frame}
	\frametitle{Reasons for studying DJ}
	\begin{itemize}
		\item DPs might provide better normative grounding
		\item Studying possibilities or hypothesis for society (Habermas like)
		\item Epistemology: conviction VS ground-based belief…
	\end{itemize}
\end{frame}

\begin{frame}
	\frametitle{Why is it not trivial?}
  “Just give her all the arguments”?
	\begin{itemize}
		\item Judgment may depend on ordering
		\item Arguments are numerous, even infinite
		\item Variants of phrasing: equivalent?
		\item Reproducibility?
		\item All this require theorizing
		\item Such claims might be falsifiable
		\item Amenable to empirical (possibly statistical) study
	\end{itemize}
\end{frame}

\section{How?}
\begin{frame}
	\frametitle{Formal approach}
	\begin{itemize}
		\item Define deliberated preference
		\item On the basis of observable reactions to arguments
		\item Define concept of a theory of DPs
		\item Define properties of theories: anonymity, falsifiability, determinacy…
		\item Analyse axiomatically
	\end{itemize}
\end{frame}

\begin{frame}[plain]
	\addtocounter{framenumber}{-1}
	\begin{center}
		\huge
		\textit{Thank you for your attention!}
	\end{center}
\end{frame}

\end{document}

\appendix
\AtBeginSection{
}

\begin{frame}[allowframebreaks]
	\frametitle{\refname}
 	\bibliography{zotero}
\end{frame}

\clearpage\pdfbookmark{License}{License}
\begin{frame}[plain]
	\frametitle{License}
	This presentation, and the associated \LaTeX{} code, are published under the \href{https://opensource.org/licenses/MIT}{MIT license}. Feel free to reuse (parts of) the presentation, under condition that you cite the author.
	
	Credits are to be given to \hrefblue{https://www.lamsade.dauphine.fr/~ocailloux/}{Olivier Cailloux}, Université Paris-Dauphine.
\end{frame}
\addtocounter{framenumber}{-1}
\end{document}

\begin{frame}
	\frametitle{Title}
	\begin{block}{Block}
		\begin{itemize}
			\item Item
		\end{itemize}
	\end{block}
\end{frame}

\begin{frame}
	\frametitle{Title}
	\begin{itemize}
		\item Item
	\end{itemize}
\end{frame}

